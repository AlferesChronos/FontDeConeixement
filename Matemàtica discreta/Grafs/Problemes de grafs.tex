\documentclass[10pt]{article}
\usepackage{PlantillaCat}


\title{Una mà de Ponts}
\author{Aleix Torres i Camps}

\begin{document}
\maketitle
\section{Nocions bàsiques}
En aquest apartat apareixen problemes relacionats amb les nocions bàsiques de connexió i distancia. A més, de problemes vinculats amb les formes matricials d'un graf. \\ \\

\textsl{\textbf{Problema 1:} El nombre de vèrtexs de grau senar en un graf $G = (V, E)$ és parell.} \\ \\
Aquest problema és el clàssic \textit{lema de les encaixades}, co\lgem orari de la següent fórmula (que va bé recordar).
$$ 
\sum_{u\in V} d(v) = 2 |E|
$$
En paraules diu que la suma dels graus del vèrtexs és igual a dos cops el nombre d'arestes. Aquest fet és evident perquè cada aresta és adjacent a exactament dos vértexs, quan sumem els graus la comptarem dues vegades. Ara, el problema ens motiva a distingir entre vèrtexs de grau senar i de grau parell. Siguin $U_1$ els vèrtexs de grau senar i $U_2$ els vértexs de grau parell ($V = U_1 \cupdot U_2$). La fórmula es pot escriure com:
$$
\sum_{u\in U_1} d(u) = 2 |E| - \sum_{u\in U_2} d(u)
$$
On a la dreta només apareixen termes parells, per tant el resultat és parell. I a l'esquerra hi ha una suma de $|U_1|$ termes senars. Sabent que aquesta ha de ser parell, n'hi ha d'haver un nombre parell. És a dir, $|U_1|$ és parell, que és el que voliem veure. \\ \\

\textsl{\textbf{Problema 2:} Qualsevol graf amb $n\geq2$ vèrtexs, en té dos del mateix grau.} \\ \\
El conjunt de possibles graus d'un graf de $n$ vèrtexs és subconjunt de $\{0,1,2,...,n-1\}$ (de cardinal $n$), ja que cada vèrtexs pot no tenir cap aresta o tenir-ne alguna fins arribar al màxim que seria ser adjacent amb els altres $n-1$ vértexs. Tot i així, en un graf no hi pot haver alhora un vèrtex de grau 0 (no és adjacent amb cap altre) i un vèrtex de grau $n-1$ (és adjacent amb tots els altres). Per tant, hi ha, com a molt, $n-1$ possibles graus diferents en un graf de $n$ vèrtexs. Aleshores, pel \textsl{Principi del Colomar}, existeixen dos vèrtexs que tenen el mateix grau, que és el que voliem demostrar. \\ \\










\textbf{Problema 2:} \\ \\
\textbf{Problema 2:} \\ \\
\textbf{Problema 2:} \\ \\
\textbf{Problema 2:} \\ \\
\textbf{Problema 2:} \\ \\
\textbf{Problema 2:} \\ \\
\textbf{Problema 2:} \\ \\
\textbf{Problema 2:} \\ \\

\textbf{Problema 2:} \\ \\
\textbf{Problema 2:} \\ \\
\textbf{Problema 2:} \\ \\
\textbf{Problema 2:} \\ \\
\textbf{Problema 2:} \\ \\
\textbf{Problema 2:} \\ \\
\textbf{Problema 2:} \\ \\
\textbf{Problema 2:} \\ \\
\textbf{Problema 2:} \\ \\


\textbf{Problema 2:} \\ \\
\textbf{Problema 2:} \\ \\
\textbf{Problema 2:} \\ \\
\textbf{Problema 2:} \\ \\
\textbf{Problema 2:} \\ \\
\textbf{Problema 2:} \\ \\
\textbf{Problema 2:} \\ \\
\textbf{Problema 2:} \\ \\
\textbf{Problema 2:} \\ \\


\textbf{Problema 2:} \\ \\
\textbf{Problema 2:} \\ \\
\textbf{Problema 2:} \\ \\
\textbf{Problema 2:} \\ \\
\textbf{Problema 2:} \\ \\
\textbf{Problema 2:} \\ \\
\textbf{Problema 2:} \\ \\
\textbf{Problema 2:} \\ \\
\textbf{Problema 2:} \\ \\


\textbf{Problema 2:} \\ \\
\textbf{Problema 2:} \\ \\
\textbf{Problema 2:} \\ \\
\textbf{Problema 2:} \\ \\
\textbf{Problema 2:} \\ \\
\textbf{Problema 2:} \\ \\
\textbf{Problema 2:} \\ \\
\textbf{Problema 2:} \\ \\
\textbf{Problema 2:} \\ \\


\end{document}
